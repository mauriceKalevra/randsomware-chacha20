% class definitions
\documentclass[12pt]{article}
\usepackage[ngerman,english]{babel}
\setlength{\parindent}{0em} 

% Packages

\usepackage[utf8]{inputenc}
\usepackage[ngerman]{babel}
\usepackage[T1]{fontenc}
\usepackage{graphicx}
\usepackage{blt}
\usepackage{lmodern}
\usepackage{tabto}
\usepackage{listings}
\usepackage{quoting} %
\usepackage{lipsum}
\usepackage[left, pagewise, edtable]{lineno}
\quotingsetup{font={itshape}, leftmargin=2em, rightmargin=0in, vskip=1ex}
\usepackage{framed} 
\usepackage{xcolor}
\usepackage{tcolorbox}
\usepackage{xcolor} 
\colorlet{shadecolor}{gray!25}
\definecolor{mshadecolor}{rgb}{0.7421875,0.7421875,0.7421875}


%bibtext


% Front page

\title{\small{WPM}\\\vspace{3mm}\Large{Advances Reverse Engineering\\\small{Projektdokumentation}}}
\author{ \small{verfasst von}\\ Moritz Rupp}
\date{Sommersemester 2023}

%Document start 
\begin{document}



\maketitle
\newpage
\tableofcontents
\newpage

\begin{abstract}
\noindent In dem Modul 'Advanced Reverse Engineering' ist es Aufgabe eine Randsomware mithilfe der Salsa20 Cipher zu entwickeln. Dabei soll eine Client-Server Infrastruktur bereitsgestellt werden, mit derer die Schadware gesteuert werden kann. Die Software wird in Python entwickelt und für Linux-Systeme ausführbar gemacht.
\end{abstract}
\section{Einführung}
Ransomware, oder auch Verschlüsselungstrojaner genannt, ist eine Malware, die das Ziel hat, den Zugriff auf bestimmte Dateien oder das gesamte Computersystem  zu verschlüsseln und anschließend Lösegeld von den Opfern zu erpressen. Dafür wird ein System mit der Schadware infiziert und anschließend anhand von starken kryptografischen Algorithmen verschlüsselt. Dadurch werden Dateien und Programme unbrauchbar gemacht. Möchte das Opfer sein System wieder entschlüsselt benötigt es den kryptografischen Schlüssel des Angreifers. Dieser wird nur gegen eine Lösegeldzahlung, meist in Form von Kryptowährungen wie Bitcoin bereitsgestellt. 
\newline


Ziel dieses Projektes ist zum einen die Implementierung der notwendigen Infrastruktur, in Form eines Client-Server Models, sowie die Erstellung der Randsomware anhand der Programmiersprache Python.
Für die eigentlich Schadaktion der Dateiverschlüsselung wird die ChaCha20 Cipher verwendet. Diese ist eine Modifikation bzw. Optimierung von Salsa20.
Zudem sollen Maßnahmen implementiert werden, die dass Analysieren der Schadware erschweren. Dafür werden Methoden der Code Obfuskation verwendet.\\
Dieses Dokument beleuchtet die Umsetzung und Implementierung der einzelnen Schritte.

\newpage
\section{Infrastruktur}
\subsection{Victim-Server}
	Auf dem Opfer-Server soll der eigentliche Randsomware Agriff stattfinden. Um diesen bereitzustellen kommen in erster Linie Virtuelle Maschienen in Frage. Diese sind jedoch sehr schwergewichtig und umstädnlich einzurichten.
	Daher wurde sich für eine Implementierung in einem Docker Container entschieden.
	Container sind eine leichtgewichtige Art von Virtualisierung auf Betriebssystemebene[1]. Dies kann genutzt werden um eine Anwendung mitsamt ihren Abhängigkeiten als eine abgeschlossene Einheit zu verpacken und zu betreiben. Dies bietet Vorteile wie Plattformunabhängigkeit und einfache Ausfürbarkeit. In diesem Fall wird ein Ubuntu Container mit einiger vorinstallierter Software wie SSH und Python aufgesetzt.\\
	Das Packet aus Anwendung und Abhängigkeiten
	nennt man auch Container-Image[2]. Dies kann anhand des Dockerfiles erzeugt werden. Das Dockerfile enthält alle Instruktionen die zur Erstellung des Images benötigt werden.  
	\begin{figure}[h]
	\caption{Dockerfile des Victim Server}
	\begin{lstlisting}[language=python, style=code]
	FROM ubuntu:latest
	RUN apt update && apt upgrade -y 
	RUN apt install  openssh-server -y
	RUN apt install sshpass -y
	RUN apt install python3-pip -y 
	RUN apt install net-tools -y 
	RUN pip install cryptography paramiko
	RUN echo "PermitRootLogin yes">etc/ssh/sshd_config
	RUN  echo 'root:root' | chpasswd
	RUN service ssh start
	EXPOSE 22
	WORKDIR /
	COPY kenndaten.py encrypt.py Passwords.txt testdaten.txt /
	COPY encryptme /encryptme
	
	CMD ["/usr/sbin/sshd", "-D"]
		\end{lstlisting}
		
		\end{figure}
\newpage
Ein Container wird stets auf einem Base Image aufgebaut. Dieses dient als Fundament für alle folge Instruktionen und enthält je nach Image ein grundlegendes Dateisystem sowie vorinstallierte Software.\\
In Zeile 1(vgl. Figure 1) wird als Base Image Ubuntu deklariert. 
Anschließend werden verschiedene Dienste wie openssh, pip sowie einige Abhängiggkeiten installiert.
In Zeile 9 wird das root Passwort für die Schnittstelle des Webservers bzw. Containers festgelegt. Die eigentliche Webanwendung wird in Zeile 13 als ein Repository durch git heruntergeladen und ins Zeile 15 installiert. In Zeile 16 wird das Kommando zum starten des SSH Dienstes und der Node Anwendung in ein Bash-Skript geschrieben, um dieses anschließend ausführbar zu machen und in Zeile 18 zu starten.\\
Mit \colorbox{mshadecolor}{\parbox{0.46\textwidth}{Docker build -t victim-http-server .}} wird aus dem Dockerfile ein Image erstellt. Alternativ kann das kompilierte Image auch über Dockerhub mit \colorbox{mshadecolor}{\parbox{0.58\textwidth}{Docker pull mauriceKalevra/ssh-node-server}} geladen werden. Dieses kann nun mit \colorbox{mshadecolor}{\parbox{0.51\textwidth}{Docker run -p 22:22 victim-http-server}} gestartet werden. Nun läuft der HTTP Server und die Webanwendung ist im Browser unter
\colorbox{mshadecolor}{\parbox{0.20\textwidth}{173.17.0.2:1337}} einsehbar. Des weiteren ist eine Admin Schnittstelle über \colorbox{mshadecolor}{\parbox{0.26\textwidth}{ssh root@173.17.0.2}} erreichbar.
\newpage
\subsection{Command and Control Server}
					
\end{document}
																					
